/home/william/Arquivos/UFLA/2018-1/Monitoria/preambulo.tex


\begin{document}
	
	% Capítulo 16.3
	\titulo{Lista 9 - Teorema Fundamental das Integrais de Linha}		
	
	\vspace{5mm}
	
	Determine se $\textbf{F}$ é ou não um campo vetorial conservativo. Se for, determine uma função $f$ tal que $\textbf{F} = \nabla f$.
	
	\begin{enumerate}
		
		% exercício 3
		\item $\textbf{F}(x,y) = (xy + y^2)\textbf{i} + (x^2 + 2xy)\textbf{j}$
		\resposta{$\nexists f$}		
		
		% exercício 4
		\item $\textbf{F}(x,y) = (y^2 - 2x)\textbf{i} + 2xy\textbf{j}$
		\resposta{\fazer}

		% exercício 5
		\item $\textbf{F}(x,y) = y^{2}e^{xy}\textbf{i} + (1+xy)e^{xy}\textbf{j}$
		\resposta{$f(x,y) = ye^{xy} + C$}

		% exercício 6
		\item $\textbf{F}(x,y) = ye^x\textbf{i} + (e^x + e^y)\textbf{j}$
		\resposta{\fazer}

		% exercício 7
		\item $\textbf{F}(x,y) = (ye^x + \sin y)\textbf{i} + (e^x + x\cos y)\textbf{j}$
		\resposta{$f(x,y) = ye^x  + x\sin y + C$}

		% exercício 8
		\item $\textbf{F}(x,y) = (3x^2 - 2y^2)\textbf{i} + (4xy + 3)\textbf{j}$
		\resposta{\fazer}

		% exercício 9
		\item $\textbf{F}(x,y) = (y^2 \cos x + \cos y)\textbf{i} + (2y\sin x - x\sin y)\textbf{j}$
		\resposta{$f(x,y) = y^2 \sin x + x \cos y + C$}
		
		% exercício 10
		\item $\displaystyle \textbf{F}(x,y) = \left(\ln y + \frac{y}{x}\right)\textbf{i} + \left(\ln x + \frac{x}{y}\right)\textbf{j}$
		\resposta{$f(x,y) = x \ln y + y \ln x + C$}

	\end{enumerate}
	
	\vspace{5mm}
	
	Determine uma função $f$ tal que $\textbf{F} = \nabla f$ e use esse resultado para calcular $\displaystyle \int_C \textbf{F} \cdot d\textbf{r}$ sobre a curva $C$ dada.
	
	\begin{enumerate}[resume]
	
		% exercício 12
		\item $\textbf{F}(x,y) = (3 + 2xy^2)\textbf{i} + 2x^{2}y\textbf{j}$, \\ $C$ é o arco da hipérbole $y = \frac{1}{x}$ de $(1,1)$ a $(4,\frac{1}{4})$
		\resposta{$f(x,y) = 3x + x^{2}y^{2} + C; \quad 9$}

		% exercício 13
		\item $\textbf{F}(x,y) = x^{2}y^{3}\textbf{i} + x^{3}y^{2}\textbf{j}$, \\ $C: \textbf{r}(t) = \langle t^3 - 2t, \, t^3 + 2t \rangle$, \, $0 \leq t \leq 1$
		\resposta{$\displaystyle f(x,y) = \frac{x^{3}y^{3}}{3} + C; \quad -9$}

		% exercício 14
		\item $\textbf{F}(x,y) = xy^{2}\textbf{i} + x^{2}y\textbf{j}$, \\ $C$ é o arco da parábola $y = 2x^2$ de $(-1,2)$ a $(2,8)$
		\resposta{\fazer}

		% exercício 15
		\item $\textbf{F}(x,y,z) = yz\textbf{i} + xz\textbf{j} + (xy + 2z)\textbf{k}$, \\ $C$ é o segmento de reta de $(1,0,-2)$ a $(4,6,3)$
		\resposta{$\displaystyle f(x,y,z) = xyz + z^2 + C; \quad 77$}

		% exercício 16
		\item $\textbf{F}(x,y,z) = (y^{2}z + 2xz^2)\textbf{i} + 2xyz\textbf{j} + (xy^2 + 2x^{2}z)\textbf{k}$, \\ $C: x = \sqrt{t}$, \, $y = t + 1$, \, $z = t^2$, \; $0 \leq t \leq 1$
		\resposta{$f(x,y,z) = xy^{2}z + x^{2}z^2 + C; \quad 5$}

		% exercício 17
		\item $\textbf{F}(x,y,z) = yze^{xz}\textbf{i} + e^{xz}\textbf{j} + xye^{xz}\textbf{k}$, \\ $C: \textbf{r}(t) = (t^2 + 1)\textbf{i} + (t^2 - 1)\textbf{j} + (t^2 - 2t)\textbf{k}$, \, $0 \leq t \leq 2$
		\resposta{$f(x,y,z) = ye^{xz} + C; \quad 4$}

		% exercício 18
		\item $\textbf{F}(x,y,z) = \sin{y}\textbf{i} + (x\cos{y} + \cos{z})\textbf{j} - y\sin{z}\textbf{k}$, \\ $C: \textbf{r}(t) = \sin{t}\textbf{i} + t\textbf{j} + 2t\textbf{k}$, \, $0 \leq t \leq \frac{\pi}{2}$
		\resposta{$f(x,y,z) = x\sin{y} + y\cos{z} + C; \quad 1 - \frac{\pi}{2}$}
	
	\end{enumerate}
	
	\vspace{5mm}
	
	Mostre que a integral de linha é independente do caminho e calcule-a.
	
	\begin{enumerate}[resume]

		% exercício 19
		\item $\displaystyle \int_C \tan y \, dx + x \sec^2 y \, dy$, \; $C$ é qualquer caminho de $(1,0)$ a $(2,\frac{\pi}{4})$
		\resposta{\fazer}

		% exercício 20
		\item $\displaystyle \int_C (1 - ye^{-x}) \, dx + e^{-x} \, dy$, \; $C$ é qualquer caminho de $(0,1)$ a $(1,2)$
		\resposta{\fazer}
	
	\end{enumerate}
		
	\vspace{5mm}	
	
	\textbf{Referência}	
	
\begin{footnotesize}
	STEWART, James. Cálculo: volume 2. 8ª ed. São Paulo, SP: Cengage Learning, 2016. ISBN 9788522125845.
\end{footnotesize}

\end{document}
