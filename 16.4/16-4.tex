\documentclass[a4paper, 12pt]{article}
\usepackage{import}
\subimport{../}{preambulo}


\begin{document}
	
	% Capítulo 16.4
	\titulo{Lista 10 - Teorema de Green}		
	
	\vspace{5mm}
	
	Use o Teorema de Green para calcular a integral de linha ao longo da curva dada, com orientação positiva.
	
	\begin{enumerate}
		
		% exercício 5
		\item $\displaystyle \oint_C ye^x \, dx + 2e^x \, dy$, \; $C$ é o retângulo com vértices $(0,0)$, $(3,0)$, $(3,4)$ e $(0,4)$\\
		\resposta{$4(e^3 - 1)$}

		% exercício 6
		\item $\displaystyle \oint_C (x^2 + y^2) \, dx + (x^2 - y^2) \, dy$, \; $C$ é o triângulo com vértices $(0,0)$, $(2,1)$ e $(0,1)$
		\resposta{$0$}

		% exercício 7
		\item $\displaystyle \oint_C (y + e^{\sqrt{x}}) \, dx + (2x + \cos{y^2}) \, dy$, \; $C$ é o limite da região englobada pelas parábolas $y = x^2$ e $x = y^2$
		\resposta{$\frac{1}{3}$}

		% exercício 8
		\item $\displaystyle \oint_C \cos y \, dx + x^{2}\sin y \, dy$, \; $C$ é o retângulo com vértices $(0,0)$, $(5,0)$, $(5,2)$ e $(0,2)$\\
		\resposta{\fazer}

		% exercício 9
		\item $\displaystyle \oint_C y^3 \, dx - x^3 \, dy$, \; $C$ é o círculo $x^2 + y^2 = 4$
		\resposta{$-24\pi$}

		% exercício 10
		\item $\displaystyle \oint_C xe^{-2x} \, dx + (x^4 + 2x^{2}y^{2}) \, dy$, \; $C$ é o limite da região entre os círculos $x^2 + y^2 = 1$ e $x^2 + y^2 = 4$\\
		\resposta{\fazer}
		
	\end{enumerate}
	
	\vspace{5mm}
	
	Use o Teorema de Green para calcular $\displaystyle \int_C \textbf{F} \cdot d\textbf{r}$, verificando a orientação da curva antes de aplicar o teorema.
	
	\begin{enumerate}[resume]

		% exercício 11
		\item $\textbf{F}(x,y) = \langle y\cos{x} - xy\sin{x}, \, xy + x\cos{x} \rangle$, \\ $C$ é o triângulo de $(0,0)$ a $(0,4)$ a $(2,0)$ a $(0,0)$
		\resposta{$-\frac{16}{3}$}

		% exercício 12
		\item $\textbf{F}(x,y) = \langle e^{-x} + y^2, \, e^{-y} + x^2 \rangle$, \quad $C$ consiste no arco da curva $y = \cos x$ de $(-\frac{\pi}{2},0)$ a $(\frac{\pi}{2},0)$ e no seguimento de reta de $(\frac{\pi}{2},0)$ a $(-\frac{\pi}{2},0)$
		\resposta{$\frac{\pi}{2}$}

		% exercício 13
		\item $\textbf{F}(x,y) = \langle y - \cos{y}, \, x\sin{y} \rangle$, \\ $C$ é o círculo $(x-3)^2 + (y+4)^2 = 4$ orientado no sentido horário
		\resposta{$4\pi$}

		% exercício 14
		\item $\textbf{F}(x,y) = \langle \sqrt{x^2 + 1}, \, \arctan{x} \rangle$, \\ $C$ é o triângulo de $(0,0)$ a $(1,1)$ a $(0,1)$ a $(0,0)$
		\resposta{$\frac{\pi - 2\ln(2)}{4}$}
	
	\end{enumerate}
		
	\vspace{5mm}	
	
	\referencia

\end{document}
