/home/william/Arquivos/UFLA/2018-1/Monitoria/preambulo.tex


\begin{document}
	
	% Capítulo 16.4
	\titulo{Lista 10 - Teorema de Green}		
	
	\vspace{5mm}
	
	Use o Teorema de Green para calcular a integral de linha ao longo da curva dada, com orientação positiva.
	
	\begin{enumerate}
		
		% exercício 5
		\item
		\resposta{}

		% exercício 6
		\item
		\resposta{}

		% exercício 7
		\item
		\resposta{}

		% exercício 8
		\item $\displaystyle \oint_C \cos y \, dx + x^{2}\sin y \, dy$, \; $C$ é o retângulo com vértices $(0,0)$, $(5,0)$, $(5,2)$ e $(0,2)$\\
		\resposta{\fazer}

		% exercício 9
		\item
		\resposta{}

		% exercício 10
		\item $\displaystyle \oint_C xe^{-2x} \, dx + (x^4 + 2x^{2}y^{2}) \, dy$, \; $C$ é o limite da região entre os círculos $x^2 + y^2 = 1$ e $x^2 + y^2 = 4$\\
		\resposta{\fazer}
		
	\end{enumerate}
	
	\vspace{5mm}
	
	Use o Teorema de Green para calcular $\displaystyle \int_C \textbf{F} \cdot d\textbf{r}$, verificando a orientação da curva antes de aplicar o teorema.
	
	\begin{enumerate}[resume]

		% exercício 11
		\item
		\resposta{}

		% exercício 12
		\item
		\resposta{}

		% exercício 13
		\item
		\resposta{}

		% exercício 14
		\item
		\resposta{}
	
	\end{enumerate}
		
	\vspace{5mm}	
	
	\textbf{Referência}	
	
\begin{footnotesize}
	STEWART, James. Cálculo: volume 2. 8ª ed. São Paulo, SP: Cengage Learning, 2016. ISBN 9788522125845.
\end{footnotesize}

\end{document}
