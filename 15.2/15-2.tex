/home/william/Arquivos/UFLA/2018-1/Monitoria/preambulo.tex


\begin{document}

	\titulo{15.2 - Integrais duplas sobre regiões gerais}		
	
	\vspace{5mm}
	
	Calcule a integral iterada.
	
	\begin{enumerate}
		
		% exercício 1
		\item $\displaystyle \int_{1}^{5} \int_{0}^{x} (8x - 2y) \, dy \, dx$ 
		\resposta{$\frac{868}{3}$}
		
		% exercício 3
		\item $\displaystyle \int_{0}^{1} \int_{0}^{y} xe^{y^{3}} \, dx \, dy$ 
		\resposta{$\frac{e-1}{6}$}
		
		% exercício 5
		\item $\displaystyle \int_{0}^{1} \int_{0}^{s^{2}} \cos(s^{3}) \, dt \, ds$ 
		\resposta{$\frac{\sin(1)}{3}$}
		
		
	\end{enumerate}
	
	\vspace{5mm}
	
	Calcule a integral dupla.
	
	\begin{enumerate}[resume]
	
		% exercício 7
		\item $\displaystyle \iint \limits_{D} \frac{y}{x^2 + 1} \, dA, \quad D = \{ (x, y) \; | \; 0 \leq x \leq 4, \, 0 \leq y \leq \sqrt{x} \}$
		\resposta{$\frac{\ln(17)}{4}$}
		
		% exercício 8
		\item $\displaystyle \iint \limits_{D} (2x + y) \, dA, \quad D = \{ (x, y) \; | \; 1 \leq y \leq 2, \; y-1 \leq x \leq 1 \}$
		\resposta{$\frac{4}{3}$}
		
		% exercício 9
		\item $\displaystyle \iint \limits_{D} e^{-y^2} \, dA, \quad D = \{ (x, y) \; | \; 0 \leq y \leq 3, \; 0 \leq x \leq y \}$
		\resposta{$\frac{1-e^{-9}}{2}$}
		
		% exercício 10
		\item $\displaystyle \iint \limits_{D} y \sqrt{x^2 - y^2} \, dA, \quad D = \{ (x, y) \; | \; 0 \leq x \leq 2, \; 0 \leq y \leq x \}$
		\resposta{$\frac{4}{3}$}

	\end{enumerate}
	
	\vspace{5mm}
	
	Calcule a integral dupla.
	
	\begin{enumerate}[resume]
	
		% exercício 17
		\item $\displaystyle \iint \limits_{D} x \cos y \, dA, \quad D$ é limitada por $y = 0, \; y = x^2, \; x = 1$
		\resposta{$\frac{1 - \cos(1)}{2}$}
		
		% exercício 18
		\item $\displaystyle \iint \limits_{D} xy^2 \, dA, \quad D$ é limitada por $x = 0$ e $x = \sqrt{1 - y^2}$
		\resposta{$\frac{2}{15}$}
		
		% exercício 19
		\item $\displaystyle \iint \limits_{D} y^2 \, dA, \quad D$ é a região triangular com vértices $(0,1)$, $(1,2)$ e $(4,1)$
		\resposta{$\frac{11}{3}$}
		
		% exercício 20
		\item $\displaystyle \iint \limits_{D} xy \, dA, \quad D$ é limitada pelo quarto de círculo $y = \sqrt{1 - x^2}$, $x \geq 0$ e pelos eixos
		\resposta{$\frac{1}{8}$}
		
		% exercício 21
		\item $\displaystyle \iint \limits_{D} (2x - y) \, dA, \quad D$ é limitada pelo círculo centrado na origem de raio 2
		\resposta{$0$}
		
		% exercício 22
		\item $\displaystyle \iint \limits_{D} y \, dA, \quad D$ é a região triangular com vértices $(0,0)$, $(1,1)$ e $(4,0)$
		\resposta{$\frac{2}{3}$}
	
	\end{enumerate}
	
	\vspace{5mm}
	
	Determine o volume do sólido dado.
	
	\begin{enumerate}[resume]
	
		% exercício 23
		\item Abaixo do plano $3x + 2y - z = 0$ e acima da região limitada pelas parábolas $y = x^2$ e $x = y^2$
		\resposta{$\frac{3}{4}$}
		
		% exercício 24
		\item Abaixo da superfície $z = 2x + y^2$ e acima da região limitada por $x = y^2$ e $x = y^3$
		\resposta{$\frac{4}{35}$}
		
		% exercício 25
		\item Abaixo da superfície $z = xy$ e acima do triângulo de vértices $(1,1)$, $(4,1)$ e $(1,2)$
		\resposta{$\frac{31}{8}$}
		
		% exercício 26
		\item Limitado pelo paraboloide $z = x^2 + y^2 + 1$ e pelos planos $x = 0$, $y = 0$, $z = 0$ e $x + y = 2$
		\resposta{$\frac{14}{3}$}
		
		% exercício 27
		\item O tetraedro limitado pelos planos coordenados e pelo plano $2x + y + z = 4$
		\resposta{$\frac{16}{3}$}
		
		% exercício 28
		\item Limitado pelo paraboloide $z = x^2 + 3y^2$ e pelos planos $x = 0$, $y = 1$, $y = x$ e $z = 0$
		\resposta{$\frac{5}{6}$}
		
		% exercício 29
		\item Limitado pelos planos coordenados e pelo plano $3x + 2y + z = 6$
		\resposta{$6$}
		
		% exercício 30
		\item Limitado pelo cilindro $y^2 + z^2 = 4$ e pelos planos $x = 2y$, $x = 0$ e $z = 0$ no primeiro octante
		\resposta{$\frac{16}{3}$}
		
		% exercício 31
		\item Limitado pelo cilindro $x^2 + y^2 = 1$ e pelos planos $y = z$, $x = 0$ e $z = 0$ no primeiro octante
		\resposta{$\frac{1}{3}$}
		
		% exercício 32
		\item Limitado pelos cilindros \; $x^2 + y^2 = r^2$ \; e \; $y^2 + z^2 = r^2$
		\resposta{$\frac{16r^3}{3}$}
	
	\end{enumerate}
			
	\vspace{5mm}	
	
	\textbf{Referência}	
	
\begin{footnotesize}
	STEWART, James. Cálculo: volume 2. 8ª ed. São Paulo, SP: Cengage Learning, 2016. ISBN 9788522125845.
\end{footnotesize}

\end{document}
