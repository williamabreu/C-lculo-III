\documentclass[a4paper, 12pt]{article}
\usepackage{import}
\subimport{../}{preambulo}


\begin{document}
	
	% Capítulo 15.9
	\titulo{Lista 7 - Mudança de variáveis em integrais múltiplas}		
	
	\vspace{5mm}
	
	Utilize a transformação dada para calcular a integral.
	
	\begin{enumerate}
		
		% exercício 15
		\item $\displaystyle \iint \limits_R (x - 3y) \, dA, \quad R$ é a região triangular com vértices $(0,0)$, $(2,1)$ e $(1,2)$; \\ $T(x,\, y) = (2u + v,\, u + 2v)$
		\resposta{$-3$}
		
		% exercício 16
		\item $\displaystyle \iint \limits_R (4x + 8y) \, dA, \quad R$ é o paralelogramo com vértices $(-1,3)$, $(1,-3)$, $(3,-1)$ e $(1,5)$; \\ $T(x,\, y) = (\frac{u-v}{4},\, \frac{v-3u}{4})$
		\resposta{$192$}
		
		% exercício 17
		\item $\displaystyle \iint \limits_R x^2 \, dA, \quad R$ é a região limitada pela elipse $9x^2 + 4y^2 = 36$; \\ $T(x,\, y) = (2u,\, 3v)$
		\resposta{$6\pi$}
		
		% exercício 18
		\item $\displaystyle \iint \limits_R (x^2 - xy + y^2) \, dA, \quad R$ é a região limitada pela elipse $x^2 - xy + y^2 = 2$; \\ $T(x,\, y) = (u\sqrt{2} - v\sqrt{\frac{2}{3}},\, u\sqrt{2} + v\sqrt{\frac{2}{3}})$
		\resposta{$\frac{4\pi\sqrt{3}}{3}$}
		
		% exercício 19
		\item $\displaystyle \iint \limits_R xy \, dA, \quad R$ é a região no primeiro quadrante limitada pelas retas $y = x$ e $y = 3x$ e pelas hipérboles $xy = 1$ e $xy = 3$; \quad $T(x,\, y) = (\frac{u}{v},\, v)$
		\resposta{$2\ln(3)$}
	
	\end{enumerate}
	
	\vspace{5mm}
	
	Resolva os problemas.
	
	\begin{enumerate}[resume]
		
		% exercício 21
		\item A Terra não é perfeitamente esférica, sendo que seu formato pode ser aproximado por um elipsoide \, $\displaystyle \frac{x^2}{a^2} + \frac{y^2}{b^2} + \frac{z^2}{c^2} = 1$, \, em que $a \approx b \approx 6378$ km e $c \approx 6356$ km. Com essas informações, estime o volume da Terra, utilizando a transformação \, $x = au$, \, $ y = bv$ \, e \, $z = cw$ \, para calcular a integral.
		\resposta{$\frac{4\pi abc}{3} \approx 1,083 \cdot 10^{12} \, \mathrm{km^3}$}
		
		% exercício 22
		\item O trabalho realizado por um motor de Carnot ideal é igual à área da região $R$ limitada por duas curvas isotérmicas \, $xy = a$ \, e \, $xy = b$ \, e duas curvas adiabáticas \, $xy^{1,4} = c$ \, e \, $xy^{1,4} = d$, \, em que $0 \leq a \leq b$ e $0 \leq c \leq d$. Calcule o trabalho realizado determinando a área de $R$.\\
		\resposta{$2,5 \cdot (b-a)\ln(\frac{d}{c})$}
		
	\end{enumerate}
	
	\vspace{5mm}
	
	Calcule a integral, efetuando uma mudança de variáveis apropriada.
	
	\begin{enumerate}[resume]
	
		% exercício 23
		\item $\displaystyle \iint \limits_R \frac{x - 2y}{3x - y} \, dA, \quad R$ é o paralelogramo limitado pelas retas \, $x - 2y = 0$, \, $x - 2y = 4$, \, $3x - y = 1$ \, e \, $3x - y = 8$
		\resposta{$\frac{8\ln(8)}{5}$}
		
		% exercício 24
		\item $\displaystyle \iint \limits_R (x + y)e^{x^2 - y^2} \, dA, \quad R$ é o retângulo limitado pelas retas \, $x - y = 0$, \, $x - y = 2$, \, $ x + y = 0$ \, e \, $x + y = 3$
		\resposta{$\frac{e^6 - 7}{4}$}
		
		% exercício 25
		\item $\displaystyle \iint \limits_R \cos \left(\frac{y-x}{y+x}\right) \, dA, \quad R$ é a região trapezoidal com vértices $(1,0)$, $(2,0)$, $(0,2)$ e $(0,1)$\\
		\resposta{$\frac{3\sin(1)}{2}$}
		
		% exercício 26
		\item $\displaystyle \iint \limits_R \sin(9x^2 + 4y^2) \, dA, \quad R$ é a região do primeiro quadrante limitada pela elipse $9x^2 + 4y^2 = 1$
		\resposta{$\left(\frac{1-\cos(1)}{24}\right)\pi$}
		
		% exercício 27
		\item $\displaystyle \iint \limits_R e^{x+y} \, dA, \quad R$ é dada pela inequação $|x| + |y| \leq 1$
		\resposta{$e - \frac{1}{e}$}
	
	\end{enumerate}
		
	\vspace{5mm}	
	
	\referencia

\end{document}
