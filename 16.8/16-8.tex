\documentclass[a4paper, 12pt]{article}
\usepackage{import}
\subimport{../}{preambulo}


\begin{document}
	
	% Capítulo 16.8
	\titulo{Lista 12 - Teorema de Stokes}		
	
	\vspace{5mm}
	
	Use o Teorema de Stokes para calcular $\displaystyle \iint_S \mathrm{rot} \, \textbf{F} \cdot d\textbf{S}$.
	
	\begin{enumerate}
		
		% exercício 2
		\item $\textbf{F}(x,y,z) = x^{2}\sin{z}\textbf{i} + y^{2}\textbf{j} + xy\textbf{k}$, \\ $S$ é a parte do paraboloide $z = 1 - x^2 - y^2$ que está em cima do plano $xy$, orientada para cima
		\resposta{$0$}

		% exercício 3
		\item $\textbf{F}(x,y,z) = ze^{y}\textbf{i} + x\cos{y}\textbf{j} + xz\sin{y}\textbf{k}$, \\ $S$ é o hemisfério $x^2 + y^2 + z^2 = 16$, $y \geq 0$, orientado na direção positiva de $y$
		\resposta{$16\pi$}

		% exercício 4
		\item $\textbf{F}(x,y,z) = x^{2}z^{2}\textbf{i} + y^{2}z^{2}\textbf{j} + xyz\textbf{k}$, \\ $S$ é a parte do paraboloide $z = x^2 + y^2$ que se encontra dentro do cilindro $x^2 + y^2 = 4$ orientado para cima
		\resposta{$0$}

		% exercício 5
		\item $\textbf{F}(x,y,z) = xyz\textbf{i} + xy\textbf{j} + x^{2}yz\textbf{k}$, \\ $S$ é formada pelo topo e pelos quatro lados, mas não pelo fundo, do cubo com vértices $(\pm 1, \pm 1, \pm 1)$, com orientação para fora
		\resposta{$0$}

		% exercício 6
		\item $\textbf{F}(x,y,z) = e^{xy}\textbf{i} + e^{xz}\textbf{j} + x^{2}z\textbf{k}$, \\ $S$ é a metade do elipsoide $4x^2 + y^2 + 4z^2 = 4$ que se situa à direita do plano $xz$, orientado na direção positiva do eixo $y$
		\resposta{$0$}

	\end{enumerate}
	
	\vspace{5mm}
	
	Use o Teorema de Stokes para calcular $\displaystyle \int_C \textbf{F} \cdot d\textbf{r}$, em que $C$ está orientado no sentido anti-horário quando visto de cima.
	
	\begin{enumerate}[resume]
	
		% exercício 7
		\item $\textbf{F}(x,y,z) =  (x + y^2)\textbf{i} + (y + z^2)\textbf{j} + (z + x^2)\textbf{k}$, \\ $C$ é o triângulo com vértices $(1,0,0)$, $(0,1,0)$ e $(0,0,1)$
		\resposta{$-1$}

		% exercício 8
		\item $\textbf{F}(x,y,z) =  yz\textbf{i} + 2xz\textbf{j} + e^{xy}\textbf{k}$, \\ $C$ é o círculo \, $x^2 + y^2 = 16$, \, z = 5
		\resposta{$80\pi$}

		% exercício 9
		\item $\textbf{F}(x,y,z) =  xy\textbf{i} + yz\textbf{j} + zx\textbf{k}$, \\ $C$ é a fronteira da parte do paraboloide $z = 1 - x^2 - y^2$ no primeiro octante
		\resposta{$-\frac{17}{20}$} 

		% exercício 10
		\item $\textbf{F}(x,y,z) =  2y\textbf{i} + xz\textbf{j} + (x + y)\textbf{k}$, \\ $C$ é a curva de interseção do plano $z = y + 2$ com o cilindro $x^2 + y^2 = 1$
		\resposta{$\pi$}
	
	\end{enumerate}
		
	\vspace{5mm}	
	
	\referencia

\end{document}
