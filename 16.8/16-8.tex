\documentclass[a4paper, 12pt]{article}
\usepackage[left=1.5cm, right=1.5cm, top=2cm, bottom=2cm]{geometry}
\usepackage[utf8]{inputenc}
\usepackage[brazil]{babel}
\usepackage{enumitem}
\usepackage{amsmath}
\usepackage{xcolor}

\pagestyle{empty}

\newcommand{\resposta}[1]{
	\hfill{{\begin{scriptsize} {\color{blue}Resposta:} \end{scriptsize}} {\color{blue}#1}}
}

\newcommand{\titulo}[1]{
	\begin{center}	
		\underline{Cálculo III}\\
		\vspace{2mm}
		\begin{large}
			\textbf{#1}	
		\end{large}
	\end{center} 
}


\begin{document}
	
	% Capítulo 16.8
	\titulo{Lista 12 - Teorema de Stokes}		
	
	\vspace{5mm}
	
	Use o Teorema de Stokes para calcular $\displaystyle \iint_S \mathrm{rot} \, \textbf{F} \cdot d\textbf{S}$.
	
	\begin{enumerate}
		
		% exercício 2
		\item
		\resposta{}

		% exercício 3
		\item
		\resposta{}

		% exercício 4
		\item $\textbf{F}(x,y,z) = x^{2}z^{2}\textbf{i} + y^{2}z^{2}\textbf{j} + xyz\textbf{k}$, \\ $S$ é a parte do paraboloide $z = x^2 + y^2$ que se encontra dentro do cilindro $x^2 + y^2 = 4$ orientado para cima
		\resposta{\fazer}

		% exercício 5
		\item
		\resposta{}

		% exercício 6
		\item
		\resposta{}

	\end{enumerate}
	
	\vspace{5mm}
	
	Use o Teorema de Stokes para calcular $\displaystyle \int_C \textbf{F} \cdot d\textbf{r}$, em que $C$ está orientado no sentido anti-horário quando visto de cima.
	
	\begin{enumerate}[resume]
	
		% exercício 7
		\item
		\resposta{}

		% exercício 8
		\item $\textbf{F}(x,y,z) =  yz\textbf{i} + 2xz\textbf{j} + e^{xy}\textbf{k}$, \\ $C$ é o círculo \, $x^2 + y^2 = 16$, \, z = 5
		\resposta{\fazer}

		% exercício 9
		\item $z = 1 - x^2 - y^2$
		\resposta{}

		% exercício 10
		\item
		\resposta{}
	
	\end{enumerate}
		
	\vspace{5mm}	
	
	/home/william/Arquivos/UFLA/2018-1/Monitoria/referencia.tex

\end{document}
