\documentclass[a4paper, 12pt]{article}
\usepackage[left=1.5cm, right=1.5cm, top=2cm, bottom=2cm]{geometry}
\usepackage[utf8]{inputenc}
\usepackage[brazil]{babel}
\usepackage{enumitem}
\usepackage{amsmath}
\usepackage{xcolor}

\pagestyle{empty}

\newcommand{\resposta}[1]{
	\hfill{{\begin{scriptsize} {\color{blue}Resposta:} \end{scriptsize}} {\color{blue}#1}}
}

\newcommand{\titulo}[1]{
	\begin{center}	
		\underline{Cálculo III}\\
		\vspace{2mm}
		\begin{large}
			\textbf{#1}	
		\end{large}
	\end{center} 
}


\begin{document}

	\titulo{15.3 - Integrais duplas em coordenadas polares}		
	
	\vspace{5mm}
	
	Calcule a integral dada, colocando-a em coordenadas polares.
	
	\begin{enumerate}
		
		% exercício 7
		\item $\displaystyle \iint \limits_{D} x^{2}y \, dA, \quad D$ é a metade superior do disco com centro na origem e raio 5
		\resposta{$\frac{1250}{3}$}
		
		% exercício 8
		\item $\displaystyle \iint \limits_{R} (2x - y) \, dA, \quad R$ é a região do primeiro quadrante limitada pelo círculo $x^2 + y^2 = 4$ e pelas retas $x = 0$ e $y = x$
		\resposta{$\frac{4(4 - 3\sqrt{2})}{3}$}
		
		% exercício 9
		\item $\displaystyle \iint \limits_{R} \sin(x^2 + y^2) \, dA, \quad R$ é a região do primeiro quadrante entre os círculos com centros na origem e raios 1 e 3
		\resposta{$\frac{[\cos(1) - \cos(9)]\pi}{4}$}
		
		% exercício 10
		\item $\displaystyle \iint \limits_{R} \frac{y^2}{x^2 + y^2} \, dA, \quad R$ é a região que fica entre os círculos \, $x^2 + y^2 = a^2$ \, e \, $x^2 + y^2 = b^2$, \, com \, $0 < a < b$
		\resposta{$\frac{(b^2 - a^2)\pi}{2}$}
		
		% exercício 11
		\item $\displaystyle \iint \limits_{D} e^{-x^2 - y^2} \, dA, \quad D$ é a região limitada pelo semicírculo $x = \sqrt{4 - y^2}$ e pelo eixo $y$ \\
		\resposta{$\frac{(1 - e^{-4})\pi}{2}$}
		
		% exercício 12
		\item $\displaystyle \iint \limits_{D} \cos(\sqrt{x^2 + y^2}) \, dA, \quad D$ é o disco com centro na origem e raio 2 \\
		\resposta{$2\pi[2\sin(2) + \cos(2) -1]$}
		
		% exercício 13
		\item $\displaystyle \iint \limits_{R} \arctan\left(\frac{y}{x}\right) \, dA, \quad R = \{ (x,y) \; | \; 1 \leq x^2 + y^2 \leq 4, \; 0 \leq y \leq x \}$
		\resposta{$\frac{3\pi^2}{64}$}		
		
		% exercício 14
		\item $\displaystyle \iint \limits_{D} x \, dA, \quad D$ é a região no primeiro quadrante que se encontra entre os círculos \, $x^2 + y^2 = 4$ \, e \, $x^2 + y^2 = 2x$
		\resposta{$\frac{16 - 3\pi}{6}$}		
				
	\end{enumerate}
			
	\vspace{5mm}	
	
	/home/william/Arquivos/UFLA/2018-1/Monitoria/referencia.tex

\end{document}
