/home/william/Arquivos/UFLA/2018-1/Monitoria/preambulo.tex


\begin{document}

	% Capítulo 15.3
	\titulo{Lista 3 - Integrais duplas em coordenadas polares}		
	
	\vspace{5mm}
	
	Calcule a integral dada, colocando-a em coordenadas polares.
	
	\begin{enumerate}
		
		% exercício 7
		\item $\displaystyle \iint \limits_{D} x^{2}y \, dA, \quad D$ é a metade superior do disco com centro na origem e raio 5
		\resposta{$\frac{1250}{3}$}
		
		% exercício 8
		\item $\displaystyle \iint \limits_{R} (2x - y) \, dA, \quad R$ é a região do primeiro quadrante limitada pelo círculo $x^2 + y^2 = 4$ e pelas retas $x = 0$ e $y = x$
		\resposta{$\frac{4(4 - 3\sqrt{2})}{3}$}
		
		% exercício 9
		\item $\displaystyle \iint \limits_{R} \sin(x^2 + y^2) \, dA, \quad R$ é a região do primeiro quadrante entre os círculos com centros na origem e raios 1 e 3
		\resposta{$\frac{[\cos(1) - \cos(9)]\pi}{4}$}
		
		% exercício 10
		\item $\displaystyle \iint \limits_{R} \frac{y^2}{x^2 + y^2} \, dA, \quad R$ é a região que fica entre os círculos \, $x^2 + y^2 = a^2$ \, e \, $x^2 + y^2 = b^2$, \, com \, $0 < a < b$
		\resposta{$\frac{(b^2 - a^2)\pi}{2}$}
		
		% exercício 11
		\item $\displaystyle \iint \limits_{D} e^{-x^2 - y^2} \, dA, \quad D$ é a região limitada pelo semicírculo $x = \sqrt{4 - y^2}$ e pelo eixo $y$ \\
		\resposta{$\frac{(1 - e^{-4})\pi}{2}$}
		
		% exercício 12
		\item $\displaystyle \iint \limits_{D} \cos(\sqrt{x^2 + y^2}) \, dA, \quad D$ é o disco com centro na origem e raio 2 \\
		\resposta{$2\pi[2\sin(2) + \cos(2) -1]$}
		
		% exercício 13
		\item $\displaystyle \iint \limits_{R} \arctan\left(\frac{y}{x}\right) \, dA, \quad R = \{ (x,y) \; | \; 1 \leq x^2 + y^2 \leq 4, \; 0 \leq y \leq x \}$
		\resposta{$\frac{3\pi^2}{64}$}		
		
		% exercício 14
		\item $\displaystyle \iint \limits_{D} x \, dA, \quad D$ é a região no primeiro quadrante que se encontra entre os círculos \, $x^2 + y^2 = 4$ \, e \, $x^2 + y^2 = 2x$
		\resposta{$\frac{16 - 3\pi}{6}$}		
				
	\end{enumerate}
	
	\vspace{5mm}
	
	Utilize a integral dupla para determinar a área da região.
	
	\begin{enumerate}[resume]
	
		% exercício 15
		\item Um laço da rosácea $r = \cos(3\theta)$
		\resposta{$\frac{\pi}{12}$}
		
		% exercício 16
		\item A região limitada por ambos os cardioides \, $r = 1 + \cos\theta$ \, e \, $r = 1 - \cos\theta$
		\resposta{$\frac{3\pi - 8}{2}$}
		
		% exercício 17
		\item A região dentro do círculo \, $(x-1)^2 + y^2 = 1$ \, e fora do círculo \, $x^2 + y^2 = 1$
		\resposta{$\frac{2\pi + 3\sqrt{3}}{6}$}
		
		% exercício 18
		\item A região dentro do cardioide \, $r = 1 + \cos\theta$ \, e fora do círculo \, $r = 3\cos\theta$
		\resposta{$\frac{\pi}{4}$}
	
	\end{enumerate}
	
	\vspace{5mm}
	
	Utilize coordenadas polares para determinar o volume do sólido dado.
	
	\begin{enumerate}[resume]
	
		% exercício 19
		\item Sob o paraboloide $z = x^2 + y^2$ e acima do disco $x^2 + y^2 \leq 25$
		\resposta{$\frac{625\pi}{2}$}
		
		% exercício 20
		\item Abaixo do cone \, $z = \sqrt{x^2 + y^2}$ \, e acima do anel \, $1 \leq x^2 + y^2 \leq 4$
		\resposta{$\frac{14\pi}{3}$}
		
		% exercício 21
		\item Abaixo do plano $2x + y + z = 4$ e acima do disco $x^2 + y^2 \leq 1$
		\resposta{$4\pi$}
		
		% exercício 22
		\item Abaixo do paraboloide $z = 18 - 2x^2 - 2y^2$ e acima do plano $xy$
		\resposta{$81\pi$}
		
		% exercício 23
		\item Uma esfera de raio $a$
		\resposta{$\frac{4\pi a^3}{3}$}
		
		% exercício 24
		\item Limitado pelo paraboloide $z = 1 + 2x^2 + 2y^2$ e pelo plano $z = 7$ no primeiro octante \\
		\resposta{$\frac{9\pi}{4}$}
		
		% exercício 25
		\item Acima do cone \, $z = \sqrt{x^2 + y^2}$ \, e abaixo da esfera \, $x^2 + y^2 + z^2 = 1$
		\resposta{$\frac{(2 - \sqrt{2})\pi}{3}$}
		
		% exercício 26
		\item Limitado pelos paraboloides \, $z = 6 - x^2 - y^2$ \, e \, $z = 2x^2 + 2y^2$
		\resposta{$6\pi$}
		
		% exercício 27
		\item Dentro tanto do cilindro \, $x^2 + y^2 = 4$ \, quanto do elipsoide \, $4x^2 + 4y^2 + z^2 = 64$ \\
		\resposta{$\frac{64\pi(8 - 3\sqrt{3})}{3}$}
	
	\end{enumerate}
			
	\vspace{5mm}	
	
	\textbf{Referência}	
	
\begin{footnotesize}
	STEWART, James. Cálculo: volume 2. 8ª ed. São Paulo, SP: Cengage Learning, 2016. ISBN 9788522125845.
\end{footnotesize}

\end{document}
