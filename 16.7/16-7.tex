\documentclass[a4paper, 12pt]{article}
\usepackage{import}
\subimport{../}{preambulo}


\begin{document}
	
	% Capítulo 16.7
	\titulo{Lista 11 - Integrais de superfície}		
	
	\vspace{5mm}
	
	Calcule a integral de superfície.
	
	\begin{enumerate}
		
		% exercício 5
		\item $\displaystyle \iint_S (x + y + z) \, dS, \quad S$ é o paralelogramo com equações paramétricas $x = u + v$, \,  $y = u - v$, \, $z = 1 + 2u + v$, \, $0 \leq u \leq 2$, \, $0 \leq v \leq 1$
		\resposta{$11\sqrt{14}$}

		% exercício 6
		\item $\displaystyle \iint_S xyz \, dS, \quad S$ é o cone com equações paramétricas $x = u \cos v$, \, $y = u \sin v$, \, $z = u$, \, $0 \leq u \leq 1$, \, $0 \leq v \leq \frac{\pi}{2}$
		\resposta{$\frac{\sqrt{2}}{10}$}

		% exercício 7
		\item $\displaystyle \iint_S y \, dS, \quad S$ é o helicoide com equação vetorial $\textbf{r}(u,v) = \langle u \cos v, \, u \sin v, \, v \rangle$, \, $0 \leq u \leq 1$, \, $0 \leq v \leq \pi$
		\resposta{$\frac{2(2\sqrt{2} - 1)}{3}$}

		% exercício 8
		\item $\displaystyle \iint_S (x^2 + y^2) \, dS, \quad S$ é a superfície com equação vetorial $\textbf{r}(u,v) = \langle 2uv, \, u^2 - v^2, \, u^2 + v^2 \rangle$, \, $u^2 + v^2 \leq 1$
		\resposta{$\pi\sqrt{2}$}

		% exercício 9
		\item $\displaystyle \iint_S x^{2}yz \, dS, \quad S$ é a parte do plano $z = 1 + 2x + 3y$ que está acima do retângulo $[0,3] \times [0,2]$
		\resposta{$171\sqrt{14}$}

		% exercício 10
		\item $\displaystyle \iint_S xz \, dS, \quad S$ é a parte do plano  $2x + 2y + z = 4$ que está no primeiro octante
		\resposta{$4$}

		% exercício 11
		\item $\displaystyle \iint_S x^{2}z^{2} \, dS, \quad S$ é a parte do cone $z^2 = x^2 + y^2$ que está entre os planos $z = 1$ e $z = 3$\\
		\resposta{\fazer}

		% exercício 12
		\item $\displaystyle \iint_S y^{2} \, dS, \quad S$ é a parte da esfera $x^2 + y^2 + z^2 = 4$ que está dentro do cilindro $x^2 + y^2 = 1$ e acima do plano $xy$
		\resposta{\fazer}

		% exercício 13
		\item $\displaystyle \iint_S z^2 \, dS, \quad S$ é a parte do paraboloide $x = y^2 + z^2$ dada por $0 \leq x \leq 1$
		\resposta{$\frac{(25\sqrt{5} + 1)\pi}{120}$}

		% exercício 14
		\item $\displaystyle \iint_S y^{2}z^{2} \, dS, \quad S$ é a parte do cone $y = \sqrt{x^2 + z^2}$ dada por $0 \leq y \leq 5$
		\resposta{$\frac{5^{6}\pi\sqrt{2}}{6}$}

		% exercício 15
		\item $\displaystyle \iint_S x \, dS, \quad S$ é a superfície $y = x^2 + 4z$, \, $0 \leq x \leq 1$, \, $0 \leq z \leq 1$
		\resposta{$\frac{21\sqrt{21} - 17\sqrt{17}}{12}$}

		% exercício 16
		\item $\displaystyle \iint_S y^{2} \, dS, \quad S$ é a parte da esfera $x^2 + y^2 + z^2 = 1$ que está acima do cone $z = \sqrt{x^2 + y^2}$
		\resposta{$\frac{(8 - 5\sqrt{2})\pi}{12}$}

		% exercício 17
		\item $\displaystyle \iint_S (x^{2}z + y^{2}z) \, dS, \quad S$ é o hemisfério $x^2 + y^2 + z^2 = 4$, \, $z \geq 0$
		\resposta{$16\pi$}

		% exercício 18
		\item $\displaystyle \iint_S (x + y + z) \, dS, \quad S$ é a parte do meio cilindro $x^2 + z^2 = 1$, $z \geq 0$, que está entre os planos $y = 0$ e $y = 2$
		\resposta{$2\pi + 4$}

		% exercício 19
		\item $\displaystyle \iint_S y \, dS, \quad S$ é a parte do paraboloide $y = x^2 + z^2$ que está dentro do cilindro $x^2 + z^2 = 4$
		\resposta{\fazer}

		% exercício 20
		\item $\displaystyle \iint_S xz \, dS, \quad S$ é o limite da região delimitada pelo cilindro $y^2 + z^2 = 9$ e pelos planos $x = 0$ e $x + y = 5$
		\resposta{\fazer}
	
	\end{enumerate}
	
	\vspace{5mm}
	
	Avalie a integral de superfície $\displaystyle \iint_S \textbf{F} \cdot d\textbf{S}$, o fluxo de $\textbf{F}$ através de $S$, para o campo vetorial $\textbf{F}$ e para a superfície orientada $S$ dada. Para superfícies fechadas, use a orientação positiva (orientado para o exterior).
	
	\begin{enumerate}[resume]

		% exercício 21
		\item $\textbf{F}(x,y,z) = ze^{xy}\textbf{i} - 3ze^{xy}\textbf{j} + xy\textbf{k}$, \\ $S$ é o paralelogramo com equações paramétricas $x = u + v$, \, $y = u - v$, \, $z = 1 + 2u + v$, \, $0 \leq u \leq 2$, \, $0 \leq v \leq 1$, com orientação ascendente
		\resposta{$4$}

		% exercício 22
		\item $\textbf{F}(x,y,z) = z\textbf{i} + y\textbf{j} + x\textbf{k}$, \\ $S$ é o helicoide com equação vetorial $\textbf{r}(u,v) = \langle u \cos v, \, u \sin v, \, v \rangle$, \, $0 \leq u \leq 1$, \, $0 \leq v \leq \pi$, com orientação ascendente
		\resposta{$\pi$}

		% exercício 23
		\item $\textbf{F}(x,y,z) = xy\textbf{i} + yz\textbf{j} + zx\textbf{k}$, \\ $S$ é a parte do paraboloide $z = 4 - x^2 - y^2$ que está acima do quadrado $0 \leq x \leq 1$, \, $0 \leq y \leq 1$, com orientação ascendente
		\resposta{$\frac{713}{180}$}

		% exercício 24
		\item $\textbf{F}(x,y,z) = -xz\textbf{i} + x\textbf{j} + y\textbf{k}$, \\ $S$ é o hemisfério $x^2 + y^2 + z^2 = 25$, $y \geq 0$, orientado na direção positiva do eixo y \\
		\resposta{\fazer}

		% exercício 25
		\item $\textbf{F}(x,y,z) = x\textbf{i} + y\textbf{j} + z^{2}\textbf{k}$, \\ $S$ é a esfera de raio 1 com centro na origem
		\resposta{\fazer}

		% exercício 26
		\item $\textbf{F}(x,y,z) = y\textbf{i} - x\textbf{j} + 2z\textbf{k}$, \\ $S$ é o hemisfério $x^2 + y^2 + z^2 = 4$, $z \geq 0$, orientado para baixo
		\resposta{$-\frac{32\pi}{3}$} 

		% exercício 27
		\item $\textbf{F}(x,y,z) = y\textbf{j} - z\textbf{k}$, \\ $S$ é formada pelo paraboloide $y = x^2 + z^2$, \, $0 \leq y \leq 1$, e pelo disco $x^2 + z^2 \leq 1$, \, $y = 1$\\
		\resposta{$0$}

		% exercício 28
		\item $\textbf{F}(x,y,z) = yz\textbf{i} + zx\textbf{j} + xy\textbf{k}$, \\ $S$ é a superfície $z = x\sin{y}$, \, $0 \leq x \leq 2$, \, $0 \leq y \leq \pi$, com orientação para cima
		\resposta{$\frac{\pi^2}{2}$}

		% exercício 29
		\item $\textbf{F}(x,y,z) = x\textbf{i} - z\textbf{j} + y\textbf{k}$, \\ $S$ é a parte da esfera $x^2 + y^2 + z^2 = 4$ no primeiro octante, com orientação para a origem
		\resposta{\fazer}

		% exercício 30
		\item $\textbf{F}(x,y,z) = xy\textbf{i} + 4x^{2}\textbf{j} + yz\textbf{k}$, \\ $S$ é a superfície $z = xe^y$, $0 \leq x \leq 1$, $0 \leq y \leq 1$, com orientação ascendente
		\resposta{\fazer}

		% exercício 31
		\item $\textbf{F}(x,y,z) = x^{2}\textbf{i} + y^{2}\textbf{j} + z^{2}\textbf{k}$, \\ $S$ é o limite do semicilindro sólido $0 \leq z \leq \sqrt{1 - y^2}$, \, $0 \leq x \leq 2$
		\resposta{$\frac{6\pi + 8}{3}$}

		% exercício 32
		\item $\textbf{F}(x,y,z) = y\textbf{i} + (z - y)\textbf{j} + x\textbf{k}$, \\ $S$ é a superfície do tetraedro com vértices $(0,0,0)$, $(1,0,0)$, $(0,1,0)$ e $(0,0,1)$
		\resposta{$-\frac{1}{6}$}
	
	\end{enumerate}
		
	\vspace{5mm}	
	
	\referencia

\end{document}
