/home/william/Arquivos/UFLA/2018-1/Monitoria/preambulo.tex


\begin{document}
	
	% Capítulo 16.7
	\titulo{Lista 11 - Integrais de superfície}		
	
	\vspace{5mm}
	
	Calcule a integral de superfície.
	
	\begin{enumerate}
		
		% exercício 5
		\item
		\resposta{}

		% exercício 6
		\item
		\resposta{}

		% exercício 7
		\item
		\resposta{}

		% exercício 8
		\item
		\resposta{}

		% exercício 9
		\item
		\resposta{}

		% exercício 10
		\item
		\resposta{}

		% exercício 11
		\item $\displaystyle \iint_S x^{2}z^{2} \, dS, \quad S$ é a parte do cone $z^2 = x^2 + y^2$ que está entre os planos $z = 1$ e $z = 3$\\
		\resposta{\fazer}

		% exercício 12
		\item $\displaystyle \iint_S y^{2} \, dS, \quad S$ é a parte da esfera $x^2 + y^2 + z^2 = 4$ que está dentro do cilindro $x^2 + y^2 = 1$ e acima do plano $xy$
		\resposta{\fazer}

		% exercício 13
		\item
		\resposta{}

		% exercício 14
		\item
		\resposta{}

		% exercício 15
		\item
		\resposta{}

		% exercício 16
		\item
		\resposta{}

		% exercício 17
		\item
		\resposta{}

		% exercício 18
		\item
		\resposta{}

		% exercício 19
		\item $\displaystyle \iint_S y \, dS, \quad S$ é a parte do paraboloide $y = x^2 + z^2$ que está dentro do cilindro $x^2 + z^2 = 4$
		\resposta{\fazer}

		% exercício 20
		\item $\displaystyle \iint_S xz \, dS, \quad S$ é o limite da região delimitada pelo cilindro $y^2 + z^2 = 9$ e pelos planos $x = 0$ e $x + y = 5$
		\resposta{\fazer}
	
	\end{enumerate}
	
	\vspace{5mm}
	
	Avalie a integral de superfície $\displaystyle \iint_S \textbf{F} \cdot d\textbf{S}$, o fluxo de $\textbf{F}$ através de $S$, para o campo vetorial $\textbf{F}$ e para a superfície orientada $S$ dada. Para superfícies fechadas, use a orientação positiva (orientado para o exterior).
	
	\begin{enumerate}[resume]

		% exercício 21
		\item
		\resposta{}

		% exercício 22
		\item
		\resposta{}

		% exercício 23
		\item
		\resposta{}

		% exercício 24
		\item $\textbf{F}(x,y,z) = -xz\textbf{i} + x\textbf{j} + y\textbf{k}$, \\ $S$ é o hemisfério $x^2 + y^2 + z^2 = 25$, $y \geq 0$, orientado na direção positiva do eixo y \\
		\resposta{\fazer}

		% exercício 25
		\item
		\resposta{}

		% exercício 26
		\item
		\resposta{}

		% exercício 27
		\item
		\resposta{}

		% exercício 28
		\item
		\resposta{}

		% exercício 29
		\item $\textbf{F}(x,y,z) = x\textbf{i} - z\textbf{j} + y\textbf{k}$, \\ $S$ é a parte da esfera $x^2 + y^2 + z^2 = 4$ no primeiro octante, com orientação para a origem
		\resposta{\fazer}

		% exercício 30
		\item $\textbf{F}(x,y,z) = xy\textbf{i} + 4x^{2}\textbf{j} + yz\textbf{k}$, \\ $S$ é a superfície $z = xe^y$, $0 \leq x \leq 1$, $0 \leq y \leq 1$, com orientação ascendente
		\resposta{\fazer}

		% exercício 31
		\item
		\resposta{}

		% exercício 32
		\item
		\resposta{}
	
	\end{enumerate}
		
	\vspace{5mm}	
	
	\textbf{Referência}	
	
\begin{footnotesize}
	STEWART, James. Cálculo: volume 2. 8ª ed. São Paulo, SP: Cengage Learning, 2016. ISBN 9788522125845.
\end{footnotesize}

\end{document}
