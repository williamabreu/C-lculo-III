\documentclass[a4paper, 12pt]{article}
\usepackage{import}
\subimport{../}{preambulo}


\begin{document}
	
	% Capítulo 16.2
	\titulo{Lista 8 - Integrais de linha}		
	
	\vspace{5mm}
	
	Calcule a integral de linha sobre a curva $C$ dada.
	
	\begin{enumerate}
		
		% exercício 1
		\item $\displaystyle \int_C y \, ds$, \quad $C: x = t^2$, \, $y = 2t$, \, $0 \leq t \leq 3$
		\resposta{$\frac{40\sqrt{10}-4}{3}$}

		% exercício 2
		\item $\displaystyle \int_C \left(\frac{x}{y}\right) \, ds$, \quad $C: x = t^3$, \, $y = t^4$, \, $1 \leq t \leq 2$
		\resposta{$\frac{73\sqrt{73}-125}{48}$}

		% exercício 3
		\item $\displaystyle \int_C xy^4 \, ds$, \quad $C$ é a metade direita do círculo $x^2 + y^2 = 16$
		\resposta{$\frac{8192}{5}$}

		% exercício 4
		\item $\displaystyle \int_C xe^y \, ds$, \quad $C$ é o segmento de reta de $(2,0)$ a $(5,4)$
		\resposta{$\frac{5(17e^4 - 5)}{16}$}

		% exercício 5
		\item $\displaystyle \int_C (x^{2}y + \sin x) \, dy$, \quad $C$ é o arco da parábola $y = x^2$ de $(0,0)$ a $(\pi,\pi^2)$
		\resposta{$\frac{\pi^2 + 6\pi}{3}$}

		% exercício 6
		\item $\displaystyle \int_C e^x \, dx$, \quad $C$ é o arco da curva $x = y^3$ de $(-1,-1)$ a $(1,1)$
		\resposta{$e - \frac{1}{e}$}

		% exercício 7
		\item $\displaystyle \int_C (x + 2y) \, dx + x^2 \, dy$, \quad $C$ consiste nos segmentos de reta de $(0,0)$ a $(2,1)$ e de $(2,1)$ a $(3,0)$
		\resposta{$\frac{5}{2}$}

		% exercício 8
		\item $\displaystyle \int_C xe^y \, dx$, \quad $C$ é o arco da curva $x = e^y$ de $(1,0)$ a $(e,1)$
		\resposta{\fazer}

		% exercício 9
		\item $\displaystyle \int_C x^{2}y \, ds$, \quad $C: x = \cos t$, \, $y = \sin t$, \, $z = t$, \, $0 \leq t \leq \frac{\pi}{2}$
		\resposta{$\frac{\sqrt{2}}{3}$}

		% exercício 10
		\item $\displaystyle \int_C y^{2}z \, ds$, \quad $C$ é o segmento de reta de $(3,1,2)$ a $(1,2,5)$
		\resposta{$\frac{107\sqrt{14}}{12}$}

		% exercício 11
		\item $\displaystyle \int_C xe^{yz} \, ds$, \quad $C$ é o segmento de reta de $(0,0,0)$ a $(1,2,3)$
		\resposta{$\frac{(e^6 - 1)\sqrt{14}}{12}$}

		% exercício 12
		\item $\displaystyle \int_C xyz^2 \, ds$, \quad $C$ é o segmento de reta de $(-1,5,0)$ a $(1,6,4)$
		\resposta{\fazer}

		% exercício 13
		\item $\displaystyle \int_C xye^{yz} \, dy$, \quad $C: x = t$, \, $y = t^2$, \, $z = t^3$, \, $0 \leq t \leq 1$
		\resposta{$\frac{2(e-1)}{5}$}

		% exercício 14
		\item $\displaystyle \int_C z \, dx + x \, dy + y \, dz$, \quad $C$ : $x = t^2$, \, $y = t^3$, \, $z = t^2$, \, $0 \leq t \leq 1$
		\resposta{\fazer}

		% exercício 15
		\item $\displaystyle \int_C z^2 \, dx + x^2 \, dy + y^2 \, dz$, \quad $C$ consiste no segmento de reta de $(1,0,0)$ a $(4,1,2)$
		\resposta{$\frac{35}{3}$}

		% exercício 16
		\item $\displaystyle \int_C (y + z) \, dx + (x + z) \, dy + (x + y) \, dz$, \quad $C$ consiste nos segmentos de reta de $(0,0,0)$ a $(1,0,1)$ e de $(1,0,1)$ a $(0,1,2)$
		\resposta{$2$}
	
	\end{enumerate}
	
	\vspace{5mm}
	
	Calcule a integral de linha $\displaystyle \int_C \textbf{F} \cdot d\textbf{r}$, em que $C$ é dada pela função vetorial $\textbf{r}(t)$.
	
	\begin{enumerate}[resume]
	
		% exercício 19
		\item $\textbf{F}(x,y) = xy^{2}\textbf{i} - x^{2}\textbf{j}$, \\ $\textbf{r}(t) = t^{3}\textbf{i} + t^{2}\textbf{j}$, \, $0 \leq t \leq 1$
		\resposta{$\frac{1}{20}$}

		% exercício 20
		\item $\textbf{F}(x,y,z) = (x + y^2)\textbf{i} + xz\textbf{j} + (y + z)\textbf{k}$, \\ $\textbf{r}(t) = t^{2}\textbf{i} + t^{3}\textbf{j} - 2t\textbf{k}$, \, $0 \leq t \leq 2$
		\resposta{$8$}

		% exercício 21
		\item $\textbf{F}(x,y,z) = \sin{x}\textbf{i} + \cos{y}\textbf{j} + xz\textbf{k}$, \\ $\textbf{r}(t) = t^{3}\textbf{i} - t^{2}\textbf{j} + t\textbf{k}$, \, $0 \leq t \leq 1$
		\resposta{$\frac{6 - 5[\sin(1) + \cos(1)]}{5}$}

		% exercício 22
		\item $\textbf{F}(x,y,z) = (x + y)\textbf{i} + (y - z)\textbf{j} + z^2\textbf{k}$, \\ $\textbf{r}(t) = t^{2}\textbf{i} + t^{3}\textbf{j} + t^{2}\textbf{k}$, \, $0 \leq t \leq 1$
		\resposta{\fazer}
	
	\end{enumerate}
		
	\vspace{5mm}	
	
	\referencia

\end{document}
