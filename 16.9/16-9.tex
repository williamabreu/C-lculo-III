\documentclass[a4paper, 12pt]{article}
\usepackage{import}
\subimport{../}{preambulo}


\begin{document}
	
	% Capítulo 16.9
	\titulo{Lista 13 - Teorema do Divergente}		
	
	\vspace{5mm}
	
	Use o Teorema do Divergente para calcular $\displaystyle \iint_S \textbf{F} \cdot d\textbf{S}$, o fluxo de $\textbf{F}$ através de $S$.
	
	\begin{enumerate}
		
		% exercício 5
		\item $\textbf{F}(x,y,z) = xye^z\textbf{i} + xy^{2}z^{3}\textbf{j} - ye^z\textbf{k}$, \\ $S$ é a superfície da caixa delimitada pelos planos coordenados e pelos planos $x = 3$, $y = 2$ e $z = 1$
		\resposta{$\frac{9}{2}$}

		% exercício 6
		\item $\textbf{F}(x,y,z) = x^{2}yz\textbf{i} + xy^{2}z\textbf{j} + xyz^{2}\textbf{k}$, \\ $S$ é a superfície da caixa delimitada pelos planos $x = 0$, $x = a$, $y = 0$, $y = b$, $z = 0$ e $z = c$, sendo $a$, $b$ e $c$ números positivos
		\resposta{$\frac{3(abc)^{2}}{4}$}

		% exercício 7
		\item $\textbf{F}(x,y,z) = 3xy^2\textbf{i} + xe^z\textbf{j} + z^3\textbf{k}$, \\ $S$ é a superfície do sólido limitado pelo cilindro $y^2 + z^2 = 1$ e pelos planos $x = -1$ e $x = 2$
		\resposta{$\frac{9\pi}{2}$}

		% exercício 8
		\item $\textbf{F}(x,y,z) = x^{2}\sin y\textbf{i} + x\cos y\textbf{j} - xz\sin y\textbf{k}$, \; $S$ é a ``esfera gorda'' $x^8 + y^8 + z^8 = 8$\\
		\resposta{$0$}

		% exercício 9
		\item $\textbf{F}(x,y,z) = xe^y\textbf{i} + (z - e^y)\textbf{j} - xy\textbf{k}$, \; $S$ é o elipsoide $x^2 + 2y^2 + 3z^2 = 4$
		\resposta{$0$}

		% exercício 10
		\item $\textbf{F}(x,y,z) = z\textbf{i} + y\textbf{j} - zx\textbf{k}$, \\ $S$ é a superfície do tetraedro limitado pelos planos coordenados e pelo plano $\displaystyle \frac{x}{a} + \frac{y}{b} + \frac{z}{c} = 1$, \; $a,b,c > 0$
		\resposta{$\frac{abc(4+a)}{24}$}

		% exercício 11
		\item $\textbf{F}(x,y,z) = (2x^3 + y^3)\textbf{i} + (y^3 + z^3)\textbf{j} + 3y^{2}z\textbf{k}$, \; $S$ é a superfície do sólido limitado pelo paraboloide $z = 1 - x^2 - y^2$ e pelo plano $xy$
		\resposta{$\pi$}

		% exercício 12
		\item $\textbf{F}(x,y,z) = (xy + 2xz)\textbf{i} + (x^2 + y^2)\textbf{j} + (xy - z^2)\textbf{k}$, \\ $S$ é a superfície do sólido limitado pelo cilindro $x^2 + y^2 = 4$ e pelos planos $z = y - 2$ e $z = 0$
		\resposta{$4\pi$}

		% exercício 13
		\item $\textbf{F}(x,y,z) = \textbf{r} |\textbf{r}|$, onde $\textbf{r} = x\textbf{i} + y\textbf{j} + z\textbf{k}$, \\ $S$ consiste no hemisfério $z = \sqrt{1 - x^2 - y^2}$ e no disco $x^2 + y^2 \leq 1$ no plano $xy$
		\resposta{$2\pi$}

		% exercício 14
		\item $\textbf{F}(x,y,z) = x^{4}\textbf{i} - x^{3}z^{2}\textbf{j} + 4xy^{2}z\textbf{k}$, \\ $S$ é a superfície do sólido limitado pelo cilindro $x^2 + y^2 = 1$ e pelos planos $z = x + 2$ e $z = 0$
		\resposta{$\frac{\pi}{5}$}
	
	\end{enumerate}
		
	\vspace{5mm}	
	
	\referencia

\end{document}
