\documentclass[a4paper, 12pt]{article}
\usepackage{import}
\subimport{../}{preambulo}


\begin{document}
	
	% Capítulo 15.7
	\titulo{Lista 5 - Integrais triplas em coordenadas cilíndricas}		
	
	\vspace{5mm}
	
	Utilize coordenadas cilíndricas para calcular a integral dada.
	
	\begin{enumerate}
		
		% exercício 17
		\item $\displaystyle \iiint \limits_{E} \sqrt{x^2 + y^2} \, dV, \quad E$ é a região que está dentro do cilindro $x^2 + y^2 = 16$ e entre os planos $z = -5$ e $z = 4$
		\resposta{$384\pi$}
		
		% exercício 18
		\item $\displaystyle \iiint \limits_{E} z \, dV, \quad E$ é limitado pelo paraboloide $z = x^2 + y^2$ e pelo plano $z = 4$
		\resposta{$\frac{64\pi}{3}$}
		
		% exercício 19
		\item $\displaystyle \iiint \limits_{E} (x + y + z) \, dV, \quad E$ é o sólido do primeiro octante que está abaixo do paraboloide $z = 4 - x^2 - y^2$
		\resposta{$\frac{128 + 40\pi}{15}$}
		
		% exercício 20
		\item $\displaystyle \iiint \limits_{E} (x - y) \, dV, \quad E$ é o sólido que está entre os cilindros $x^2 + y^2 = 1$ e $x^2 + y^2 = 16$, acima do plano $xy$ e abaixo do plano $z = y + 4$
		\resposta{$-\frac{255\pi}{4}$}
		
		% exercício 21
		\item $\displaystyle \iiint \limits_{E} x^2 \, dV, \quad E$ é o sólido que está dentro do cilindro $x^2 + y^2 = 1$, acima do plano $z = 0$ e abaixo do cone $z^2 = 4x^2 + 4y^2$
		\resposta{$\frac{2\pi}{5}$}
		
		% exercício 22
		\item $\displaystyle \iiint \limits_{E} x \, dV, \quad E$ é limitado pelos planos $z = 0$ e $z = x + y + 5$ e pelos cilindros $x^2 + y^2 = 4$ e $x^2 + y^2 = 9$
		\resposta{$\frac{65\pi}{4}$}
	
	\end{enumerate}
	
	\vspace{5mm}
	
	Utilize coordenadas cilíndricas para determinar o volume do sólido.	
	
	\begin{enumerate}[resume]
	
		% exercício 23
		\item Limitado pelo cone $z = \sqrt{x^2 + y^2}$ e abaixo da esfera $x^2 + y^2 + z^2 = 2$
		\resposta{$\frac{4\pi(\sqrt{2} - 1)}{3}$}
		
		% exercício 24
		\item Entre o paraboloide $z = x^2 + y^2$ e a esfera $x^2 + y^2 + z^2 = 2$
		\resposta{$\frac{(8\sqrt{2} - 7)\pi}{6}$}
		
		% exercício 25
		\item Limitado pelo paraboloide $z = 24 - x^2 - y^2$ e pelo cone $z = 2\sqrt{x^2 + y^2}$
		\resposta{$\frac{512\pi}{3}$}
		
	\end{enumerate}
		
	\vspace{5mm}	
	
	\referencia

\end{document}
