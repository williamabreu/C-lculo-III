/home/william/Arquivos/UFLA/2018-1/Monitoria/preambulo.tex


\begin{document}

	% Capítulo 15.6
	\titulo{Lista 4 - Integrais triplas}		
	
	\vspace{5mm}
	
	Calcule a integral iterada.
	
	\begin{enumerate}
		
		% exercício 3
		\item $\displaystyle \int_{0}^{2} \int_{0}^{z^2} \int_{0}^{y-z} (2x - y) \, dx \, dy \, dz$
		\resposta{$\frac{16}{15}$}
		
		% exercício 4
		\item $\displaystyle \int_{0}^{1} \int_{y}^{2y} \int_{0}^{x+y} 6xy \, dz \, dx \, dy$
		\resposta{$\frac{23}{5}$}
		
		% exercício 5
		\item $\displaystyle \int_{0}^{\frac{\pi}{2}} \int_{0}^{y} \int_{0}^{x} \cos(x+y+z) \, dz \, dx \, dy$
		\resposta{$-\frac{1}{3}$}
	
	\end{enumerate}
	
	\vspace{5mm}
	
	Calcule a integral tripla
	
	\begin{enumerate}[resume]
	
		% exercício 9
		\item $\displaystyle \iiint \limits_{E} y \, dV, \quad E = \{ (x,y,z) \; | \; 0 \leq x \leq 3, \; 0 \leq y \leq x, \; x-y \leq z \leq x+y \}$
		\resposta{$\frac{27}{2}$}
		
		% exercício 10
		\item $\displaystyle \iiint \limits_{E} e^{\frac{z}{y}} \, dV, \quad E = \{ (x,y,z) \; | \; 0 \leq y \leq 1, \; y \leq x \leq 1, \; 0 \leq z \leq xy \}$
		\resposta{$\frac{3e - 7}{6}$}
		
		% exercício 11
		\item $\displaystyle \iiint \limits_{E} 2x \, dV, \quad E = \{ (x,y,z) \; | \; 0 \leq y \leq 2, \; 0 \leq x \leq \sqrt{4 - y^2}, \; 0 \leq z \leq y \}$
		\resposta{$4$}
		
		% exercício 12
		\item $\displaystyle \iiint \limits_{E} xy \, dV, \quad E$ é limitado pelos cilindros parabólicos $y = x^2$ e $x = y^2$ e pelos planos $z = 0$ e $z = x + y$
		\resposta{$\frac{3}{28}$}
		
		% exercício 13
		\item $\displaystyle \iiint \limits_{E} 6xy \, dV, \quad E$ está abaixo do plano $z = 1 + x + y$ e acima da região do plano $xy$ limitado pelas curvas \, $y = \sqrt{x}$, \, $y = 0$ \, e \, $x = 1$
		\resposta{$\frac{65}{28}$}
		
		% exercício 14
		\item $\displaystyle \iiint \limits_{E} (x - y) \, dV, \quad E$ é limitado pelas superfícies \, $z = x^2 - 1$, \, $z = 1 - x^2$, \, $y = 0$ \, e \, $y = 2$
		\resposta{$-\frac{16}{3}$}
		
		% exercício 15
		\item $\displaystyle \iiint \limits_{T} y^2 \, dV, \quad T$ é o tetraedro sólido com vértices $(0,0,0)$, $(2,0,0)$, $(0,2,0)$ e $(0,0,2)$ \\
		\resposta{$\frac{8}{15}$}
		
		% exercício 16
		\item $\displaystyle \iiint \limits_{T} xz \, dV, \quad T$ é o tetraedro sólido com vértices $(0,0,0)$, $(1,0,1)$, $(0,1,1)$ e $(0,0,1)$ \\
		\resposta{$\frac{1}{144}$}

		% exercício 17
		\item $\displaystyle \iiint \limits_{E} x \, dV, \quad E$ é limitado pelo paraboloide $x = 4y^2 + 4z^2$ e pelo plano $x = 4$
		\resposta{$\frac{16\pi}{3}$}
		
		% exercício 18
		\item $\displaystyle \iiint \limits_{E} z \, dV, \quad E$ é limitado pelo pelo cilindro $y^2 + z^2 = 9$ e pelos planos $x = 0$, $y = 3x$ e $z = 0$ no primeiro octante
		\resposta{$\frac{27}{8}$}
	
	\end{enumerate}
	
	\vspace{5mm}
	
	Use a integral tripla para determinar o volume do sólido dado.
	
	\begin{enumerate}[resume]
	
		% exercício 19
		\item O tetraedro limitado pelos planos coordenados e pelo plano $2x + y + z = 4$
		\resposta{$\frac{16}{3}$}
		
		% exercício 20
		\item O sólido limitado pelos paraboloides $y = x^2 + z^2$ e $y = 8 - x^2 - z^2$
		\resposta{$16\pi$}
		
		% exercício 21
		\item O sólido limitado pelo cilindro parabólico $y = x^2$ e pelos planos $z = 0$ e $y + z = 1$
		\resposta{$\frac{8}{15}$}
		
		% exercício 22
		\item O sólido limitado pelo cilindro $x^2 + z^2 = 4$ e pelos planos $y = -1$ e $y + z = 4$
		\resposta{$20\pi$}
	
	\end{enumerate}
			
	\vspace{5mm}	
	
	\textbf{Referência}	
	
\begin{footnotesize}
	STEWART, James. Cálculo: volume 2. 8ª ed. São Paulo, SP: Cengage Learning, 2016. ISBN 9788522125845.
\end{footnotesize}

\end{document}
