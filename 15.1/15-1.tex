\documentclass[a4paper, 12pt]{article}
\usepackage[left=1.5cm, right=1.5cm, top=2cm, bottom=2cm]{geometry}
\usepackage[utf8]{inputenc}
\usepackage[brazil]{babel}
\usepackage{enumitem}

\newcommand{\resposta}[1]{\hfill{{\begin{scriptsize} Resposta: \end{scriptsize}} #1}}


\begin{document}

	\thispagestyle{empty}

	\begin{large} 
	\begin{center}		
		\textbf{15.1 - Integrais duplas sobre retângulos}		
	\end{center}
	\end{large} 
	
	\vspace{5mm}
	
	Calcule a integral iterada.
	
	\begin{enumerate} %[resume]
		
		% exercício 15
		\item $\displaystyle \int_{1}^{4} \int_{0}^{2} (6x^{2}y + 2x) \, dy \, dx$ 
		\resposta{$222$}
		
		% exercício 16
		\item $\displaystyle \int_{0}^{1} \int_{0}^{1} (x + y)^2 \, dx \, dy$ 
		\resposta{$\frac{7}{6}$}
		
		% exercício 17
		\item $\displaystyle \int_{0}^{1} \int_{1}^{2} (x + e^{-y}) \, dx \, dy$ 
		\resposta{$\frac{5}{2} - e^{-1}$}
		
		% exercício 18
		\item $\displaystyle \int_{0}^{\frac{\pi}{6}} \int_{0}^{\frac{\pi}{2}} (\sin x + \sin y) \, dy \, dx$ 
		\resposta{$(\frac{8 - 3\sqrt{3}}{12})\pi$}
		
		% exercício 19
		\item $\displaystyle \int_{0}^{2} \int_{0}^{\frac{\pi}{2}} (x \sin y) \, dy \, dx$ 
		\resposta{$2$}
		
	\end{enumerate}
		
	\vspace{5mm}	
	
	\textbf{Referência}	
	
	\begin{footnotesize}
		STEWART, James. Cálculo: volume 2. 8ª ed. São Paulo, SP: Cengage Learning, 2016.	ISBN 9788522125845.
	\end{footnotesize}

\end{document}
